
\begin{center}
\addtolength{\leftskip}{-6cm} % increase (absolute) value if needed
\addtolength{\rightskip}{-6cm}
\begin{longtable}{
    >{\raggedright}b{3cm}
    >{\raggedleft}b{3cm}
    >{\raggedleft}b{3cm}
    >{\raggedleft}b{3cm}
    >{\raggedleft}b{3cm}
    >{\PBS\raggedleft}b{3cm}
}
\label{tbl:table5}
\caption{Factors predicting the number of overprescribed medicines in the medication review group. Results from the full and univariate analysis are presented for the model with patient-related factors and the model with medication-related factors. All models were generalized linear models with quasi-Poisson distribution (log link). The full models included all medicine groups or all patient-related baseline characteristics without near-zero variance. Confidence limits and P-values are not adjusted for multiple comparisons. The exponentiated β is the ratio between the predicted number of overprescribed medicines for patients “having” the factor versus patients not “having” the factor (with all other factors being equal). The reference level (i.e. what is meant by not “having” the factor) is listed in parentheses for binary and categorical factors. For example, patients referred from the geriatric department have 74\% (or 26\% fewer) the number of overprescribed medicines compared with patients referred from their GP. For continuous variables, an increase of one is used, e.g. patients with a Drug Burden Index of 1 have 115\% (or 15\% more) overprescribed medicines compared with patients with a Drug Burden Index of 0, and patients with a Drug Burden Index of 2 have 15\% more overprescribed medicines compared with patients with a Drug Burden Index of 1 (or 115\% × 115\% = 132\% overprescribed medicines compared with a patient with a Drug Burden Index of 0).}
\toprule
\multirow{2}{=}{} &
  \multirow{2}{=}{\textbf{Descriptive   statistics}} &
  \multicolumn{2}{m{6cm}}{\textbf{Full model}} &
  \multicolumn{2}{m{6cm}}{\textbf{Univariate model}} \\ \cmidrule(lr){3-6} 

                                           &             & \textbf{Exponentiated  \boldmath$\beta$ (95\% CI)} & \textbf{\textit{P} value} & \textbf{Exponentiated \boldmath$\beta$ (95\% CI)} & \textbf{\textit{P} value}        \\ \cmidrule(r){1-6}
{\textbf{Model:   Baseline characteristics}} & & & & & \\
Not motivated for medicine changes (ref.   Motivated for medicine changes), No. (\%) &
  39 (19.9\%) &
  0.74 (0.59 to 0.92) &
  .009 &
  0.72 (0.55 to 0.93) &
  .016 \\
Referred from the geriatric department   (ref. Referred from GP), No. (\%) &
  59 (30.1\%) &
  0.75 (0.58 to 0.96) &
  .024 &
  0.78 (0.63 to 0.97) &
  .026 \\
Age in years   (continuous), median (IQR) &
  80 (74 to 85) &
  1.02 (1.00 to 1.03) &
  0.009 &
  1.003 (0.99 to 1.02) &
  .63 \\
No. of medicines at baseline, median (IQR) &
  12 (10 to 14) &
  1.08 (1.05 to 1.11) &
  \textless .001 &
  1.09 (1.07 to 1.12) &
  \textless .001 \\
Drug Burden Index, median (IQR) &
  0.5 (0 to 1) &
  1.15 (1.03 to 1.29) &
  .015 &
  1.26 (1.13 to 1.40) &
  \textless .001 \\
\textbf{Model: Medicine   groups} &
  \multirow{2}{*}{} &
  \multirow{2}{*}{} &
  \multirow{2}{*}{} &
  \multirow{2}{*}{} &
  \multirow{2}{*}{} \\
(ref. Medicine   group not used), No. (\%) &             &                             &         &                           &                \\
Propulsives                                & 15 (7.7\%)  & 1.42 (1.02 to 1.97)         & .038    & 1.73 (1.30 to 2.26)       & \textless .001 \\
~~Metoclopramid                              & 15          &                             &         &                           &                \\
Iron bivalent, oral preparations           & 28 (14.3\%) & 1.51 (1.14 to 2.01)         & .005    & 1.40 (1.07 to 1.80)       & .013           \\
~~Ferrous fumarate                           & 16          &                             &         &                           &                \\
~~Ferrous sulphate                           & 11          &                             &         &                           &                \\
~~Ferrous tartrate                           & 1           &                             &         &                           &                \\
Other antidepressants                      & 26 (13.3\%) & 1.52 (1.13 to 2.04)         & .006    & 1.66 (1.29 to 2.11)       & \textless .001 \\
~~Mirtazapin                                 & 14          &                             &         &                           &                \\
~~Venlafaxin                                 & 5           &                             &         &                           &                \\
~~Duloxetin                                  & 4           &                             &         &                           &                \\
~~Mianserin                                  & 2           &                             &         &                           &                \\
~~Agomelatin                                 & 1           &                             &         &                           &                \\
Propionic acid derivatives                 & 19 (9.7\%)  & 1.54 (1.14 to 2.07)         & .005    & 1.47 (1.11 to 1.91)       & .006           \\
~~Ibuprofen                                  & 18          &                             &         &                           &                \\
~~Dexibuprofen                               & 1           &                             &         &                           &                \\
Drugs for urinary frequency and   incontinence &
  14 (7.1\%) &
  1.97 (1.37 to 2.79) &
  \textless{}.001 &
  1.36 (0.95 to 1.87) &
  .076 \\
~~Mirabegron                                 & 8           &                             &         &                           &                \\
~~Tolterodine                                & 4           &                             &         &                           &                \\
~~Solifenacin                                & 1           &                             &         &                           &                \\
~~Trospium                                   & 1           &                             &         &                           &                \\ \bottomrule
\end{tabular}
\end{center}
\end{table}