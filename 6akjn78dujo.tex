\begin{table}[!h]
\label{tbl:table3}
\caption{Top 10 medicines with the highest absolute difference in the proportion of discontinuations between the groups. Only medicines prescribed to at least 10 patients at baseline are included.}
\begin{center}
\addtolength{\leftskip} {-4cm} % increase (absolute) value if needed
\addtolength{\rightskip}{-4cm}
\begin{tabular}{
        p{5cm}
        >{\raggedleft}p{4cm}
        >{\raggedleft}p{4cm}
        >{\PBS\raggedleft}p{2cm}
}
\toprule
\textbf{Medicine} &
  \multicolumn{2}{p{6cm}}{\textbf{Discontinued at first visit, No. discontinued / No. at   baseline = \% discontinued}} &
  \textbf{Absolute difference, No. (\%)} \\ \cmidrule(lr){2-3}
                            & \textbf{Medication review   (n = 196)} & \textbf{Usual care (n = 212)} &         \\ \cmidrule(lr){1-4}
Metoclopramide              & 11 / 15 = 73\%                & 1 / 12 = 8\%         & 10 (65) \\
Acetylsalicylic   acid      & 20 / 48 = 42\%                & 2 / 47 = 4\%         & 18 (37) \\
Simvastatin                 & 18 / 48 = 38\%                & 2 / 58 = 3\%         & 16 (34) \\
Formoterol and   budesonide & 5 / 15 = 33\%                 & 0 / 16 = 0\%         & 5 (33)  \\
Zopiclone                   & 23 / 59 = 39\%                & 4 / 54 = 7\%         & 19 (32) \\
Tiotropium   bromide        & 4 / 14 = 29\%                 & 0 / 17 = 0\%         & 4 (29)  \\
Allopurinol                 & 3 / 11 = 27\%                 & 0 / 12 = 0\%         & 3 (27)  \\
Quinine                     & 9 / 14 = 64\%                 & 6 / 16 = 38\%        & 3 (27)  \\
Citalopram                  & 4 / 18 = 22\%                 & 0 / 20 = 0\%         & 4 (22)  \\
Tramadol                    & 18 / 37 = 49\%                & 8 / 30 = 27\%        & 10 (22) \\ \bottomrule
\end{tabular}
\end{center}
\end{table}